
%%%%\section{Mu3e}

The $\mu \rightarrow eee$ decay is a charged lepton flavor violating process strongly suppressed in the Standard Model. New physics mediated either via virtual loop or three diagrams can enhance these rates to values accessible by the next generation of experiments. An interesting feature of this process is 
the possibility to determine the chirality of New Physics, should it be observed with sufficient statistics~\cite{Okada:1999zk}. The current limit ${\cal B}(\mu^+\to e^+e^-e^+) < 1\times 10^{-12}$ has been set by the SINDRUM experiment at PSI~\cite{Bellgardt:1987du}. The Mu3e experiment~\cite{Blondel:2013ia} has been proposed to improve this bound by four orders of magnitude.

We present a detector concept to search for $\mu \rightarrow eee$ decay using the FastSim simulation package. The experimental setup  consists of a compact tracker made of 6 layers of cylindrical silicon detectors, each composed of 50 $\mu$m thick silicon sensors mounted on 50 $\mu$m of kapton. The target is formed of two hollow cones, having each a length and radius of 5 cm and 1 cm, respectively. Contrary to Mu3e, we consider an active target made of silicon pixel detectors, assuming a pixel size of 50 $\mu$m by 50 $\mu$m. Although not included, a time-of-flight system should be installed as well, providing a time measurement with a resolution of 250~ps or better. The apparatus is displayed in Fig.~\ref{Fig::mu3e}, together with a simulated $\mu^+ \rightarrow e^+e^-e^+$ event.

We generate $\mu^+ \rightarrow e^+e^-e^+$ events according to phase space, and constrain the tracks to originate from the same pixel in the active target. To further improve the resolution, we require the probability of the constrained fit to be greater than 1\%, and the reconstructed muon momentum less than $1 ~\Mev$. The absolute value of the cosine of the polar angle of each electron must also be less than 0.9. The resulting $e^+e^-e^+$ invariant mass distribution is shown in Fig.~\ref{Fig::mu3e}, and peaks sharply at the muon mass. We extract the resolution by fitting this spectrum with a double-sided Crystal Ball function (a Gaussian with power-law tails on both sides). The Gaussian resolution is found to be 0.3~\Mev\ for a signal efficiency of 27\%.  

To achieve a single event sensitivity at the level of $5\times\sim 10^{-18}$ after a 3-year run, a stop muon rate of the order of $8\times 10^{9}$ is needed. For comparison, the estimated stop muon rate at Mu3e with the HiMB beam is expected to be $2\times 10^{9}$~\cite{Blondel:2013ia}.

For the purpose of estimating background contributions, we define a signal window as $ 104.9 < m_{eee} < 106.5 ~\Mev$, containing approximately 90\% of the signal. The accidental background arise from $\mu \rightarrow e^+e^-e^+ \nu\bar\nu$ events where the two neutrinos carry almost no energy. We estimate its contribution to be about 7.5 events by convolving the branching fraction with the resolution function and integrating in the signal region. However, this background depends strongly on the tail resolution, and small improvements translate into large background reductions. For example, decreasing the thickness of the silicon sensors and the supporting kapton structure by 20\% (40\%) reduces the background down to $\sim 4$ ($\sim 1$) events. 

\begin{figure}[htb]
\begin{center}
\includegraphics[width=0.45\textwidth]{ChargedLeptons/Figures/event.pdf}\includegraphics[width=0.45\textwidth]{ChargedLeptons/Figures/resoFit.pdf}
\end{center}
\caption{Left: Display of the experimental setup, together with a simulated $\mu^+ \rightarrow e^+e^-e^+$ event. 
Right:  The $e^+e^-e^+$ invariant mass distribution after all selection criteria are applied fitted by a 
sum of two Gaussian functions.}
\label{Fig::mu3e}
\end{figure}

We consider accidental backgrounds produced by the combination of a Michel decay and a radiative Michel decay (2M$\gamma$ decays), or three simultaneous Michel decays (3M decays), where one positron is misreconstructed or produces an electron by interacting within the detector. In both cases, we assume that
the decays occur within the same pixel in the active target and within the same time window. This yields position and time suppression factors $\delta S = 7.8\times 10^{-7}$ and $\delta t = 2.5\times 10^{-10}$, respectively. The background rate is given by:
%
$$N_{2M\gamma} = {R_\mu}^2 \delta S \delta t {B(\mu^+ \rightarrow e^+ \nu_e \bar\nu_\mu)}^2 B(\mu^+ \rightarrow e^+ \nu_e \bar\nu_\mu \gamma) P(\gamma \rightarrow e^+ e^-)  P_\mu  \simeq 0.33 P_\mu$$
$$N_{3M} = {R_\mu}^3(\delta S)^2 {B(\mu^+ \rightarrow e^+ \nu_e \bar\nu_\mu)}^3 (\delta t)^2 P_\mu \simeq 0.02 P_\mu$$
%
where $P(\gamma \rightarrow e^+ e^-)\sim 0.18\%$ is the probability of photon conversion in the target and $P_\mu$ denotes the probability to reconstruct a muon candidate after all selection criteria are applied. We estimate the factors $P_\mu \sim {\cal O}(10^{-8})$ for 2M$\gamma$ decays and $P_\mu \sim {\cal O}(10^{-9})$ for 3M decays with our simulation. For a 3-year run and with a rate of $8\times 10^{9}$ stopped muons per second, both backgrounds are found to be less than an event.

In summary, we outline the requirements needed to improve by an order of magnitude the projected $\mu \rightarrow eee$ 
sensitivity of the mu3e experiment. We study a similar design, with the addition on an active target instead of a passive one. Assuming a 3-year run, a rate of $8\times 10^{9}$ stopped muons in the target per second would be required. Relatively modest improvements on the resolution are also needed to maintain the irreducible background at an appropriate level, while an active target proves to be essentially in the reduction of accidental 
backgrounds. 







