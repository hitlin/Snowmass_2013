

If we assume that neutrino mass is generated by a seesaw mechanism \cite{Minkowski:1977sc}, we can see effects in CLFV if the seesaw scale is low~\cite{de
Gouvea:2007uz}. It is perhaps more natural that the seesaw mechanism
is realized at a very high energy scale, $M_R \sim 10^{10} - 10^{14}$
GeV. In this case there can be significant CLFV provided that there is
some new physics at the TeV scale. Below we briefly mention two such
scenarios, SUSY and Randall-Sundrum warped extra dimensions (RS).

Implementing the seesaw mechanism within the context of SUSY leads to
a new source of CLFV. In a momentum range  between $M_R$ and  $M_{\rm
Pl}$, where $M_{\rm Pl}$ is the fundamental Planck scale, the
right-handed neutrinos are active and their Dirac
Yukawa couplings with the lepton doublets induce flavor
violation among the sleptons.  The sleptons must have masses of order
TeV or less, if SUSY is to solve the hierarchy problem, and they carry
information on flavor violation originating from the seesaw.
Specifically, the squared masses of the sleptons would receive flavor
violating contributions given by
\begin{equation}
(m_{\tilde{l}_{L}}^{2})_{ij} \simeq -\frac{1}{8\pi^{2}}
(Y_{\nu}^\dagger Y_{\nu})_{ij} (3 m_{0}^{2} + |A_{0}|^{2})\ln\left({M_{\rm
Pl}\over M_{R}}\right)
\label{CL:SUSY-LFV}
\end{equation}
where $Y_\nu$ is the Dirac Yukawa coupling of the neutrinos, and $m_0$
and $A_0$ are SUSY breaking mass parameters of order $100$ GeV.  In
SUSY GUTs, even without neutrino masses, there is an independent
contribution to CLFV, originating from the grouping of quarks and
leptons in the same GUT multiplet.   The squared masses of
the right-handed sleptons would receive contributions to CLFV in this momentum range from the GUT
scale particles that are active, given by
\begin{equation}
(m_{\tilde{e}_{R}}^{2})_{ij} \simeq -\frac{3}{8\pi^{2}}
V_{3i}V_{3j}^{*}|Y_t|^{2}(3 m_{0}^{2}  +
|A_{0}|^{2})\ln\left(\frac{M_{\rm Pl}}{M_{{\rm GUT}}}\right).
\label{CL:GUT_LFV}
\end{equation}
Here $V_{ij}$ denote the known CKM quark mixing matrix elements, and
$Y_t$ is the top quark Yukawa coupling.  Unlike
Eq.~(\ref{CL:SUSY-LFV}), which has some ambiguity since $Y_\nu$ is not
fully known, the CLFV contribution from Eq.~(\ref{CL:GUT_LFV}) is
experimentally determined, apart from the SUSY parameters. 

We next consider RS models with
bulk gauge fields and fermions.   In these models, our universe
 is localized on one (ultraviolet) membrane of a multidimensional 
 space while the Higgs field is localized on a different (infrared) membrane. 
  Each particle has a wave function that is localized near the Higgs 
  membrane for heavy particles or near our membrane 
  for light particles. Thus localization of different wave functions between
   the membranes  generates flavor.
For a given fermion mass spectrum, there are only two free parameters,
 an energy scale to set the Yukawa couplings and a length scale of 
 compactification that sets the level of Kaluza-Klein  (KK) excitations.  The two scales can be accessed using a combination of tree induced CLFV processes that occur in $\mu N\rightarrow e N$ or $\mu\rightarrow 3e$ and loop induced interactions such as $\mu\rightarrow e\gamma$ \cite{hep-ph/0606021,arXiv:1004.2037}. The amplitude of loop-induced
flavor-changing decays, such as $\mu\to e \gamma$, is given by a
positive power of the Yukawa and a negative power of the KK
scale.  Tree-level flavor-changing diagrams, on the other hand, come
from four-fermion interactions whose flavor-changing vertices come
from the non-universal profile of an intermediate KK gauge boson. This
non-universality is an effect of electroweak symmetry breaking so that
the flavor-changing part of the KK gauge boson profile is localized
near the IR brane and the size of flavor-changing effects depend on
the size of the zero mode fermion profile towards the IR
brane. However, in order to maintain the Standard Model fermion
spectrum the zero-mode fermion profiles must be pushed away from the
Higgs vacuum expectation value on the IR brane as the anarchic Yukawa scale is
increased. Thus the tree-level flavor changing amplitudes go like a
negative power of the anarchic Yukawa scale. For a given KK scale,
experimental constraints on lepton flavor-changing processes at tree
and loop level thus set lower and upper bounds on the Yukawa scale,
respectively.

A version of minimal flavor violation exists in RS models where the new
 scales have a very small effect on low energy flavor changing processes.  
 It was noted in \cite{hep-ph/0606021} and \cite{Agashe:2004cp} that
certain flavor changing diagrams are suppressed in the RS scenario
because the particular structure of zero mode wave functions and
Yukawa matrices is the same as the zero mode mass terms induced by
electroweak symmetry breaking. When passing to the physical basis
of light fermions these processes are also nearly diagonalized, or
\textit{aligned}, and off-diagonal elements of these transitions are
suppressed. These flavor-changing processes are not completely zero 
since the fermion bulk masses are an additional flavor spurion
in these theories. In other words, the $U(3)^3$ lepton flavor 
symmetry is not restored in the limit where the Yukawa terms vanish. 
%
The full one-loop calculation of $\mu\to e \gamma$ in Randall-Sunrdum
models including these misalignment effects and a proof of finiteness 
was performed in \cite{arXiv:1004.2037}.

