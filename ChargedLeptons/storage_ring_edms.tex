
%%%%%%%%%%%%%%%%%%%%%%%%%%%%%%%%%%%
% For all our labels we use :cl: (cl for charged leptons), like
% eq:cl:mueg
% That way we will not have problems when combining it

%%%%%%%%%%%%%%%%%%%%%%%%%%%%%%%%%%%%%%%%%%%%%%%%%%%%%%%%%%%
%%%%%%%%%%%%%%%%%%%%%%%%%%%%%%%%%%%%%%%%%%%%%%%%%%%%%%%%%%%
%%%%%%%%%%%%%%%%%%%%%%%%%%%%%%%%%%%%%%%%%%%%%%%%%%%%%%%%%%%
%%%%%%%%%%%%%%%%%%%%%%%%%%%%%%%%%%%%%%%%%%%%%%%%%%%%%%%%%%%



As detailed above, the equation for the spin precession frequency of a charged particle in a storage ring is
\begin{equation}
\vec{\omega}_{a\eta}= \vec \omega_a + \vec \omega_\eta = -
 \frac {Qe}  {m}
\left[
a \vec{B}
- \left( a - \left(  \frac {m} {p} \right)^2  \right)
 \frac {\vec{\beta} \times \vec{E}} {c} \right] -  \eta \frac {Qe}{2m}
 \left[ \frac {\vec{E}} {c}  +  \vec{\beta} \times \vec{B} \right] .
 \label{eq:omegaa-edm2}
\end{equation}
The discussion above focused on experiments operating at the magic momentum, $p = \sqrt{a }/m$ that cancels the effect of the focusing $E$ field.  The precession frequency is then by far dominated by the $aB$ term.  
The key to extracting sensitivity to the EDM term $\eta$ is to find ways of reducing or eliminating the motion due to the magnetic term $a$.  

The first method is to use a magnetic storage ring, such as the E821/E989 storage ring, to extract a 
limit on the muon EDM.  In the muon rest frame, the muon sees a strong motional electric field
 pointing towards the center of the ring adding a small horizontal component to the precession 
 frequency vector that tilts the rotation plane.  For a positive EDM, when the spin is pointing into the 
 ring it will have a negative vertical component and when the spin is pointing to the outside of the ring 
 it will have a positive vertical component.  Since the positrons are emitted along the spin direction, 
 this asymmetry maps into the positron decay angle. Since the asymmetry is maximized when the spin 
 and momentum are perpendicular, the angular asymmetry is 90 degrees out of phase with the $g-2$ precession frequency.  
Searches for this asymmetry have been used to set limits on the muon EDM both at the CERN and Brookhaven $g-2$ experiments. 

 A number of the Fermilab Muon $g-2$ detector stations will
be instrumented with straw chambers to measure the decay positron
tracks. With this instrumentation, a simultaneous EDM
measurement can be made during the $a_\mu$ data collection,
improving on the  Brookhaven muon EDM~\cite{Bennett08-edm}
 limit by up to two orders of magnitude down to
$\sim 10^{-21}\,  e \cdot {\rm cm}$.   The primary detector element of the J-PARC muon $g-2$ proposal  is a silicon tracker that will provide
 decay angle information for all tracks and expects similar improvements.

Going beyond this level for the muon will require a dedicated EDM experiment that
uses
the ``frozen spin'' method~\cite{Farley04,Roberts2010}.
 The idea is to operate a
muon storage ring off of the  magic momentum and to use a radial electric field
to cancel the $\omega_a$ term in Eq.~\ref{eq:omegaa-edm1},
 the $g-2$ precession.  The  electric field needed to freeze the spin is
$E \simeq aBc\beta\gamma^2$.
Once the spin is frozen, the EDM will cause a steadily increasing
out-of-plane motion of the spin vector. One stores polarized muons in a ring
with detectors above and below the storage region and forms the asymmetry
(up - down)/(up + down).  To reach a sensitivity of $10^{-24}e \cdot {\rm cm}$ would
require $\sim 4 \times 10^{16}$ recorded events~\cite{Farley04}.
 Preliminary discussions have begun on a frozen spin experiment
using the ~1000 kW beam power available at the Project X rare process campus.
  
It is possible to make a direct measurement of the electron EDM using a storage ring analogous to way $g-2$ is extracted for the muon.  
Here, the key to removing the magnetic precession is to use an electrostatic storage ring so that any spin precession can be attributed to an EDM.  Stray radial magnetic fields 
would lead to a false signal.  The effects of these can be measured and controlled by using counter rotating beams. Several years of 
R$\&$D have already been invested into this technique either for searching for a proton, deuteron, or muon EDM.  Fortuitously, the ratio
 of the $g-2$ value to the mass of the electron is very similar to this ratio for the proton.  Many systematic effects scale with this ratio so 
 that many studies already performed for the proton can be used for the electron.  Conversely, performing a storage ring EDM 
 experiment on the electron would be an excellent test bed for a (more expensive) proton EDM experiment.
 
 Also fortuitous for the electron is that its magic momentum is 15 MeV, requiring a relatively small and inexpensive storage ring.  The technology for the polarized source,
  electrostatic magnets, and beam position monitors all seem to be available.  Concepts for the polarimeter are still being 
  developed and are expected to be the limiting factor in the ultimate sensitivity of the experiment.  A polarimeter with high 
  analyzing power would most likely lead to a sensitivity comparable with model independent extractions of the electron 
  EDM from atoms and molecules.



