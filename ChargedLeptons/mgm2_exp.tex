
% test for first committ

%\subsubsection{Dipole Moments}
%\label{cl:dipole-moments}

%%%%%%%%%%%%%%%%%%%%%%%%%%%%%%%%%%%
% For all our labels we use :cl: (cl for charged leptons), like
% eq:cl:mueg
% That way we will not have problems when combining it

%%%%%%%%%%%%%%%%%%%%%%%%%%%%%%%%%%%%%%%%%%%%%%%%%%%%%%%%%%%
%%%%%%%%%%%%%%%%%%%%%%%%%%%%%%%%%%%%%%%%%%%%%%%%%%%%%%%%%%%
%%%%%%%%%%%%%%%%%%%%%%%%%%%%%%%%%%%%%%%%%%%%%%%%%%%%%%%%%%%
%%%%%%%%%%%%%%%%%%%%%%%%%%%%%%%%%%%%%%%%%%%%%%%%%%%%%%%%%%%




Measurements of the magnetic and electric dipole moments make use of the
torque on a dipole in an external field, $\vec \tau = \vec\mu \times
\vec B + \vec d \times \vec E$. All muon MDM experiments except the original
Nevis ones used polarized muons in flight, and
 measured the rate at which the spin turns relative to the momentum,
$\vec \omega_a =\vec \omega_S - \vec \omega_C$, when a
 beam of polarized muons is injected into a magnetic field.
The resulting frequency, assuming that $\vec \beta \cdot \vec B = 0$,
 is given
by~\cite{Thomas26,Bargmann59}
\begin{equation}
\vec{\omega}_{a\eta}= \vec \omega_a + \vec \omega_\eta = -
 \frac {Qe}  {m}
\left[
a \vec{B}
- \left( a - \left(  \frac {m} {p} \right)^2  \right)
 \frac {\vec{\beta} \times \vec{E}} {c} \right] -  \eta \frac {Qe}{2m}
 \left[ \frac {\vec{E}} {c}  +  \vec{\beta} \times \vec{B} \right] .
 \label{eq:omegaa-edm1}
\end{equation}
Important features of this equation are the motional magnetic and
electric fields:
$\vec \beta \times \vec E$ and $\vec \beta \times \vec B$.


The E821 Collaboration working at the
Brookhaven AGS used an electric quadrupole field
to provide vertical focusing in the storage ring, and shimmed the magnetic
field to 1 ppm uniformity on average.  The storage ring was operated
 at the `` $g-2$'' momentum, $p_{ g-2} = 3.094$~GeV/c,
($\gamma_{ g-2}= 29.3$),
so that $a_\mu = (m/p)^2$ and the electric field did not
contribute to $\omega_a$.
They obtained\cite{Bennett06}
\begin{equation}
  a_\mu^{(\mathrm{E821})} = 116\,592\,089(63) \times
  10^{-11}~~\mbox{(0.54\,ppm)}
\end{equation}
 The final uncertainty of
0.54~ppm consists of a 0.46~ppm statistical component and a 0.28~ppm
systematic
component.  

The present limit on the muon EDM also comes from E821~\cite{Bennett08-edm}
\begin{equation}
d_\mu = (0.1 \pm 0.9) \times 10^{-19} e  \cdot {\rm cm}; \
 \vert d_\mu \vert < 1.9 \times 10^{-19}  e  \cdot {\rm cm}\ (95\%\ {\rm C.L.})\, ,
\label{muedm-result}
\end{equation}
so the EDM contribution to the precession is very small.  In the muon $g-2$
experiments, the motional electric field dominates the $\omega_\eta$ term,
which means that $\vec \omega_a$ and $\vec \omega_\eta$ are orthogonal.
The presence of an EDM in the  $g-2$ momentum experiments has two effects:
the measured frequency is the quadrature sum of the two frequencies,
 $\omega = \sqrt{\omega_a^2 + \omega_\eta^2}$, and the EDM causes a tipping of
 the plane of precession, by an angle
 $\delta = \tan ^{-1}[ \eta \beta/(2a_\mu)]$. This tipping results in
 in an up-down oscillation of the decay
 positrons relative to the midplane of the storage ring with frequency
$\omega_a$ {\it out of phase by} $\pi/2$ with the $a_\mu$ precession.

The E989 collaboration
at Fermilab will move the E821 muon storage ring to Fermilab, and
will use the  $g-2$ momentum technique to measure $a_{\mu^+}$.
 New detectors and electronics, and a
beam handling scheme that increases the stored muon rate per hour
 by a factor of 6
over E821 will be implemented.  The goal is at least 21 times the
statistics of E821, and a factor of four overall uncertainty reduction, with
equal systematic and statistical uncertainties of $\pm 0.1$ ppm.

The scope of Project X includes 50-200kW of beam power at 8 GeV,
about three to fifteen times the beam power of E989.  This large step in beam
power could be used to measure $g-2$ for negative muons,
and provide muon beams with lower emittance thereby reducing
experimental systematics. 

Given the high impact of the E821 result and the 
crucial role the value of $g-2$ plays in interpreting energy frontier results, 
it is imperative to have a second measurement with at least equal 
precision but with a complementary approach to the measurement. 
An alternate approach planned for J-PARC~\cite{JPARC-Lg2} uses a much lower muon
energy, and does not use the  $g-2$ momentum technique. A surface muon beam
produced by  the low energy Booster is brought to rest in an aerogel
target, where muonium (the $\mu^+ e^-$ atom) is formed.  The muonium
is ionized by a powerful laser which produces a very slow muon beam with
extremely small emittance. This low emittance beam is then accelerated by a
linac to 300 MeV, and injected into a  $\sim 1$~m diameter
solenoidal magnet with point to point
uniformity of $\pm 1$ ppm, approximately 100 times better than at the Brookhaven experiment.  
The average uniformity is expected to be known to better then 0.1 ppm.  The decays are detected 
by a full volume tracker consisting of an array of silicon
detectors.  This provides time, energy, and decay angle information for every positron, maximizing
 the sensitivity to separate the  $g-2$ and EDM precession frequencies.  The expected   $g-2$ sensitivity 
 is comparable to the
 Fermilab experiment but will have very different systematic uncertainties and the combined results 
 from the two experiments should bring the precision to below the 100 ppb level.



