In contrast to muon CLFV searches, in which a single dedicated experiment is required for a given decay,  $\tau$~lepton CLFV searches are conducted using the large data sets collected in comprehensive $e^+e^-$ or hadron collider experiments. The relative theoretical parameter reach of $\mu$ and $\tau$ decay experiments is model-dependent, and thus comparisons of limits or observations in the two cases can serve to distinguish between models.
Tests with taus can be more powerful on an event-by-event basis than those using muons, since the large $\tau$ mass greatly decreases 
Glashow-Iliopoulos-Maiani (GIM) suppression, correspondingly increasing new physics partial widths (typically by a factor of $\geq 500$ in $\cal{B}(\tau \rightarrow \mu \gamma)$ or $e \gamma$ {\it vs.} $\cal{B}(\mu \rightarrow$$e \gamma$)).  The difficulty is that one can typically produce $\sim 10^{11}$ muons per second, while the samples from \babar\ and Belle collected over the past decade together total $\sim 10^{10}$ events.  

The new generation of super $B$ or $\tau$/$c$ factories, \cite{superkekb},\cite{Blinov:2009zzd},\cite{tc} promise to extend the
experimental reach in $\tau$ decays to levels that sensitively
probe new physics in the lepton sector. Since CLFV is severely suppressed in the Standard Model, CLFV $\tau$ decays
are especially clean probes for BSM Physics
effects.  The
super flavor factories can access $\tau$ CLFV decay rates two orders of magnitude smaller than current limits for the cleanest channels
({\it e.g.}, $\tau\to 3\ell$), and at least one order of magnitude smaller for other
modes that have irreducible backgrounds, such as $\tau\to \ell\gamma$. Super flavor factories thus have a sensitivity for CLFV decays that directly confronts many BSM models. 

Polarized beams at an $e^+e^-$ collider can provide further
experimental advantages.   Belle II at \hbox{SuperKEKB} will not have a polarized beam, but both the proposed BINP and Tor Vergata $\tau$/$c$ factories will have polarized electron beams.  Polarization of the taus thus produced provides several advantages. It allows reduction of backgrounds in certain
CLFV decay modes, as well as providing sensitive new
observables that increase precision in other important measurements, including
searches for $C\!P$ violation in $\tau$ production and decay, the measurement
of $g-2$ of the $\tau$, and the search for a $\tau$ EDM.  Preliminary studies indicate that polarization improves the sensitivity on these quantities by a factor of two to three. Should the CLFV decay $\tau \rightarrow 3\ell$ be found, a study of the Dalitz plot of the polarized $\tau$ decay can determine the Lorentz structure of the CLFV coupling.

The provision of polarization requires a polarized electron gun, 
a lattice that supports transverse
polarization at the desired CM energy, a means of interchanging transverse polarization in the
ring and longitudinal polarization at the interaction point and a means of monitoring
the polarization, typically a Compton polarimeter to monitor the
backscattering of circularly polarized laser light. Achieving useful longitudinal polarization at the interaction point requires sufficiently long depolarization time of the machine lattice, which is highly dependent on the details of the lattice and the beam energy.

Provision of a
polarized positron beam is difficult and expensive; it is generally
also regarded as unnecessary, as most of the
advantages of polarization for the measurements cited above can be
accomplished with a single polarized beam.

\begin{figure}[htb]

\begin{center}
\includegraphics[width=8cm]{ChargedLeptons/Figures/Belle-tau.pdf}
\smallskip
\caption{\label{CL:Belle}Extrapolation of the 90\% upper limit sensitivity of Belle-II (open symbols) from existing limits (filled symbols). For $\tau\rightarrow \mu\gamma$, which has irreducible backgrounds, the limit scales as $1/{\sqrt{\!\int{\!\cal{L}}dt}}$. For $\tau\rightarrow \mu\mu\mu$, which is essentially background-free, the limit scales as $1/{\int{\!\cal{L}}dt}$.}
\end{center}
\end{figure}

The sensitivity of $\tau$ CLFV searches at SuperKEKB has been estimated by extrapolating from current CLEO, Belle and \babar\  
limits (see Figure~\ref{CL:Belle}). The optimization of search sensitivities depends on the size
of the sample as well as on the sources of background. For SuperKEKB, the
extrapolation for the (largely background-free) $\tau\to\ell\ell\ell$
modes assumes $1/{\mathcal L}$ scaling up to 5 ab$^{-1}$; that for $\tau\to\ell\gamma$
modes scales as $1/{\sqrt{\mathcal L}}$. 
The expected sensitivities for several modes are shown for the Belle II experiment in Table~\ref{tab:LFVExptSensitivities-BelleII}~\cite{Abe:2010sj}.  

\begin{table}[!b]
  \caption{
    \label{tab:LFVExptSensitivities-BelleII}
    Expected $90\%$ CL upper limits
    on $\tau\to\mu\gamma$, $\tau\to \mu\mu\mu$, and $\tau\to \mu\eta$
    with $5 \ {\rm ab}^{-1}$ and  $50 \ {\rm ab}^{-1}$ data sets from Belle II and  Super KEKB.
  }
  \begin{center}
    \begin{tabular}{lll}
      \hline \hline
Process & $5\ {\rm ab}^{-1}$ & $50\ {\rm ab}^{-1}$  \\
      \hline
      $\BR(\tau \to \mu\,\gamma) \rule{0pt}{2.6ex}$ &  $10 \times 10^{-9}$ &  $3 \times 10^{-9}$  \\
      $\BR(\tau \to \mu\, \mu\, \mu)$ &  $3 \times 10^{-9}$ & $1 \times 10^{-9}$  \\
      $\BR(\tau \to \mu \eta)$             &  $5 \times 10^{-9}$ & $2\times 10^{-9}$    \\
 \hline\hline
    \end{tabular}
  \end{center}
\end{table}


These CLFV sensitivities directly confront a large variety of new
physics models. Of particular interest is the correlation between
$\tau$ CLFV branching ratios such as $\tau\to \mu\gamma$ and $\tau\to e
\gamma$, as well as the correlation with $\mu\to e \gamma$ and the
$\mu\to e$ conversion rate, all of which are diagnostic of particular
models.  A polarized electron beam potentially allows the possibility of determining the helicity structure of CLFV couplings from Dalitz plot analyses of, for example, $\tau \to 3\ell$ decays.

The experimental situation at a $\tau/c$ factory is somewhat different. The luminosity of the proposed projects is $10^{35}$cm$^{-2}$s$^{-1}$, a factor of eight below the eventual SuperKEKB luminosity. The $\tau$ production cross section is, however, larger: $\sigma_{\tau\bar{\tau}}(3.77\  \gev)/\sigma_{\tau\bar{\tau}}(10.58\  \gev)=3$, and both have a polarized electron beam. In addition, while a Super $B$ factory is likely to spend the bulk of its running time at the $\Upsilon(4S)$, a $\tau/c$ factory will take data more evenly throught the accessible energy range.  A study for the BINP machine~\cite{bobrov}, with 1.5 ab$^{-1}$ at 3.686 GeV,
3.5 ab$^{-1}$ at 3.770 GeV, and 2.0 ab$^{-1}$ at 4.170 GeV, 
corresponding to  $2.5\times 10^{10}$ produced $\tau$ pairs, quotes a 90\% confidence level limit on $\cal{B}(\tau\rightarrow\mu\gamma)$= $3.3\times 10^{-10}$, provided the detector has $\mu/\pi$ rejection of a factor of 30. This is nearly an order of magnitude improvement over the SuperKEKB expectation at 50 ab$^{-1}$.

%\subsubsection{LHCb}

The LHC is a prolific source of $\tau$ leptons with an expected
production cross section of about 0.1mb, the majority coming from 
decays of $D_s$ mesons and $B$ hadrons.  
The LHCb experiment~\cite{Alves:2008zz} can profit from this large $\tau$ lepton
production rate thanks to its forward geometry and the flexible
trigger system. The LHCb collaboration has published a search for the
decay $\tau \to \mu\mu\mu$ with 1/fb of data at 7 TeV which  obtained a 90\% CL
limit of $8 \times 10^{-8}$~\cite{Aaij:2013fia}. The total dataset recorded in run 1 of the
LHC is 3/fb at an energy of 7 and 8 TeV.

The expected future sensitivity in this channel can be estimated
conservatively as scaling with the square root of the luminosity. By the end
of run 2 of the LHC (2018), an additional 5/fb are expected to be
collected at a beam energy of 13TeV. The sensitivity using this
dataset should be competitive with the current Belle sensitivity. 
The upgraded LHCb detector will start data taking in 2018 and is
expected to collect a dataset of 50/fb at 14TeV.
The sensitivity in the $\tau \to \mu\mu\mu$ search using this dataset is $8\times
10^{-9}$, assuming conservatively a scaling with the root of the
luminosity. In constrast, when assuming improvements on the analysis
strategy, the optimistic assumption of linear scaling can be made. The
sensitivity on $\tau \to \mu\mu\mu$ decays would then be $7 \times 10^{-10}$ with the
upgraded LHCb experiment. 

As the upgraded LHCb experiment will have a very efficient trigger
system also for softer hadrons, LFV $\tau$ decays with one or several
hadrons in the final state can reach sensitivities that are only slightly reduced
with respect to the purely muonic decay. 


